% Required to support mathematical unicode
\usepackage[warnunknown, fasterrors, mathletters]{ucs}
\usepackage[utf8x]{inputenc}

% Always typeset math in display style
\everymath{\displaystyle}

% Use a larger font size
\usepackage[fontsize=11.5pt]{scrextend}

% Standard mathematical typesetting packages
\usepackage{amsfonts, amsthm, amsmath, amssymb}
\usepackage{mathtools}  % Extension to amsmath
\usepackage{thmtools}
\usepackage[framemethod=TikZ]{mdframed}


% Symbol and utility packages
\usepackage{cancel, textcomp}
\usepackage[mathscr]{euscript}
\usepackage[nointegrals]{wasysym}

% Extras
\usepackage{physics}  % Lots of useful shortcuts and macros
\usepackage{tikz}
\usepackage{tikz-cd}  % For drawing commutative diagrams easily
\usepackage{color}  % Add some colour to life
\usepackage{xcolor}
\usepackage{microtype}  % Minature font tweaks
\usepackage[margin=1.0in]{geometry} 
\usepackage{calc}

% For Front Matter
\usepackage[
	colorlinks=true,
	breaklinks,
	allcolors=black,
	hidelinks
	]{hyperref}
\usepackage[nameinlink]{cleveref}
% Common shortcuts
\def\mbb#1{\mathbb{#1}}
\def\mfk#1{\mathfrak{#1}}

\def\bN{\mbb{N}}
\def\bC{\mbb{C}}
\def\bR{\mbb{R}}
\def\bQ{\mbb{Q}}
\def\bZ{\mbb{Z}}

% Sometimes helpful macros
\newcommand{\func}[3]{#1\colon#2\to#3}
\newcommand{\vfunc}[5]{\func{#1}{#2}{#3},\quad#4\longmapsto#5}
\newcommand{\floor}[1]{\left\lfloor#1\right\rfloor}
\newcommand{\ceil}[1]{\left\lceil#1\right\rceil}
\newcommand{\ti}[1]{\textit{#1}}
\newcommand{\tb}[1]{\textbf{#1}}
\newcommand{\un}[1]{\underline{#1}}
\newcommand{\hs}[1]{\hspace{#1}}
\newcommand{\vs}[1]{\vspace{#1}}
\newcommand{\fn}[1]{\footnote{#1}}

% Thoerems 

\mdfdefinestyle{mdbluebox}{%
	linewidth=1pt,
	skipabove=12pt,
	innerbottommargin=9pt,
	skipbelow=2pt,
	nobreak=false,
	linecolor=blue,
	backgroundcolor=blue!5,
}

\mdfdefinestyle{mdredbox}{%
	linewidth=0.5pt,
	skipabove=12pt,
	frametitleaboveskip=5pt,
	frametitlebelowskip=0pt,
	skipbelow=2pt,
	frametitlefont=\bfseries,
	innertopmargin=4pt,
	innerbottommargin=8pt,
	nobreak=false,
	linecolor=red,
	backgroundcolor=orange!5,
}
\mdfdefinestyle{mdgreenbox}{%
	skipabove=8pt,
	linewidth=2pt,
	rightline=false,
	leftline=true,
	topline=false,
	bottomline=false,
	nobreak=false,
	linecolor=green,
	backgroundcolor=green!5,
}
\mdfdefinestyle{mdblackbox}{%
	skipabove=8pt,
	linewidth=3pt,
	rightline=false,
	leftline=true,
	topline=false,
	bottomline=false,
	nobreak=false,
	linecolor=black,
	backgroundcolor=black!5!gray!5,
}
\declaretheoremstyle[headfont=\bfseries\sffamily, bodyfont=\normalfont, mdframed={style=mdgreenbox} ]{thmgreenbox}
\declaretheoremstyle[headfont=\bfseries\sffamily, bodyfont=\normalfont, mdframed={style=mdredbox} ]{thmredbox}
\declaretheoremstyle[headfont=\bfseries\sffamily, bodyfont=\normalfont, mdframed={style=mdbluebox}]{thmbluebox}
\declaretheoremstyle[headfont=\bfseries\sffmaily, bodyfont=\normalfont, mdframed={style=mdblackbox}]{thmblackbox}

\declaretheorem[numberwithin=chapter, style=thmgreenbox, name=Definition]{definition}
\declaretheorem[sibling=definition, style=thmredbox, name=Corollary]{corollary}
\declaretheorem[sibling=definition, style=thmredbox, name=Proposition]{prop}
\declaretheorem[numberwithin=,style=thmblackbox, name=]{Proof}
\declaretheorem[sibling=definition, style=thmbluebox, name=Remark]{remark}
\declaretheorem[sibling=definition, style=thmredbox, name=Theorem]{theorem}
\declaretheorem[sibling=definition, style=thmredbox, name=Lemma]{lemma}
\declaretheorem[sibling=definition, style=thmbluebox,  name=Example]{eg}
\declaretheorem[sibling=definition, style=thmbluebox,  name=Nonexample]{noneg}

